%\documentclass[draft, a4paper,openany,oneside,12pt]{report}
\documentclass[a4paper,openany,oneside,12pt]{report}
%\usepackage{helvet}
%\renewcommand{\familydefault}{\sfdefault}

\usepackage[T1]{fontenc}
\usepackage{textcomp}
\usepackage[spanish]{babel} % espanol
\usepackage[utf8]{inputenc} % acentos sin codigo
\renewcommand{\shorthandsspanish}{}
\addto{\captionsspanish}{\def\chaptername{}} % quita la palabra capítulo de los encabezados

\usepackage[hidelinks]{hyperref}
\hypersetup{colorlinks=true,
linkcolor=blue,
urlcolor=blue,
linkcolor=blue,
citecolor=red}
\usepackage[apaciteclassic]{apacite}
\usepackage{titlesec, blindtext, color}
\definecolor{gray75}{gray}{0.75}
\newcommand{\hsp}{\hspace{20pt}}
\titleformat{\chapter}[hang]{\Huge\bfseries}{\thechapter\hsp\textcolor{gray75}{|}\hsp}{0pt}{\Huge\bfseries}
\usepackage{graphicx} % graficos
\usepackage{caption}
\usepackage{subcaption}
\usepackage{pdfpages}
\usepackage{setspace}
\usepackage{fancyhdr}
\usepackage{eurosym}
\usepackage{verbatim}
\usepackage{xcolor}
\usepackage{shadow} %Para usar caja con sombra \shabox{}
\usepackage{lettrine}						% Para hacer letras capitales.
\usepackage{rotating}
\usepackage{wasysym} % Para simbolos adicionales
\usepackage{import}
\usepackage{multirow}
\usepackage{svg}
\usepackage{amssymb}
\usepackage{siunitx}
\usepackage{setspace} % Interlineados
\usepackage[toc,page]{appendix}
\usepackage{listings}
\usepackage{sclang-prettifier}
\usepackage{float}
\usepackage{dirtree}
\usepackage{anyfontsize} % Para cambiar el tamaño de letra arbritrariamente: {\fontsize{0.1}{60}\selectfont Foo} https://osl.ugr.es/CTAN/macros/latex/contrib/anyfontsize/anyfontsize.pdf

\usepackage{verbatim}

\lstset{basicstyle=\ttfamily\footnotesize,breaklines=true}
\lstset{frame=bottomline}
\lstset{literate=
	{á}{{\'a}}1 {é}{{\'e}}1 {í}{{\'i}}1 {ó}{{\'o}}1 {ú}{{\'u}}1
	{Á}{{\'A}}1 {É}{{\'E}}1 {Í}{{\'I}}1 {Ó}{{\'O}}1 {Ú}{{\'U}}1
	{à}{{\`a}}1 {è}{{\`e}}1 {ì}{{\`i}}1 {ò}{{\`o}}1 {ù}{{\`u}}1
	{À}{{\`A}}1 {È}{{\'E}}1 {Ì}{{\`I}}1 {Ò}{{\`O}}1 {Ù}{{\`U}}1
	{ä}{{\"a}}1 {ë}{{\"e}}1 {ï}{{\"i}}1 {ö}{{\"o}}1 {ü}{{\"u}}1
	{Ä}{{\"A}}1 {Ë}{{\"E}}1 {Ï}{{\"I}}1 {Ö}{{\"O}}1 {Ü}{{\"U}}1
	{â}{{\^a}}1 {ê}{{\^e}}1 {î}{{\^i}}1 {ô}{{\^o}}1 {û}{{\^u}}1
	{Â}{{\^A}}1 {Ê}{{\^E}}1 {Î}{{\^I}}1 {Ô}{{\^O}}1 {Û}{{\^U}}1
	{œ}{{\oe}}1 {Œ}{{\OE}}1 {æ}{{\ae}}1 {Æ}{{\AE}}1 {ß}{{\ss}}1
	{ű}{{\H{u}}}1 {Ű}{{\H{U}}}1 {ő}{{\H{o}}}1 {Ő}{{\H{O}}}1
	{ç}{{\c c}}1 {Ç}{{\c C}}1 {ø}{{\o}}1 {å}{{\r a}}1 {Å}{{\r A}}1
	{€}{{\EUR}}1 {£}{{\pounds}}1
}


\sisetup{load-configurations = abbreviations}
\newcommand{\grad}{\hspace{-2mm}$\phantom{a}^{\circ}$}
\clubpenalty=20000	% para que no haya líneas viudas en el texto.
\widowpenalty=20000 % para que no haya huérfanas
% indica a Latex contabilizar el nivel de gerarquía de texto hasta el cuarto nivel;
\setcounter{secnumdepth}{4}
% indica al comando de generación de tabla de contenido que incluya el cuarto nivel:
\setcounter{tocdepth}{4}


%% Title format
\begin{comment}
\titleformat
{\chapter} % command
[display] % shape
{\bfseries\LARGE} % format
{Capítulo \ \thechapter} % label
{0.5ex} % sep
{
	\rule{\textwidth}{1pt}
	\vspace{1ex}
	\centering
} % before-code
[
\vspace{-0.5ex}%
\rule{\textwidth}{0.3pt}
] % after-code
\end{comment}

\renewcommand{\appendixname}{Anexos}
\renewcommand{\appendixtocname}{Anexos}
\renewcommand{\appendixpagename}{Anexos}

\newcommand{\className}{SynthiGME}
\newcommand{\appName}{Synthi \textsc{gme}}
\newcommand{\version}{1.0.0}

% Keywords command
\providecommand{\keywords}[1]
{
	\noindent
	\small	
	\textbf{\textit{Keywords---}} #1
}

% Keywords command (en español)
\providecommand{\palabrasClave}[1]
{
	\noindent
	\small	
	\textbf{\textit{Palabras clave---}} #1
}

%%%%%%%%%%%%%%%%%%%%%%%%%%%%%%%%%%%%%%%%%%%%%%%%%%

\begin{document}
\bibliographystyle{apacite}
 \selectlanguage{spanish}
	\import{./portada/}{portada.tex}
	\newpage
	\mbox{}
%	\thispagestyle{empty} % para que no se numere esta página
	\setcounter{page}{2}
	\thispagestyle{plain}
	\pagestyle{fancy}

\onehalfspace

% Página en blanco
\vfill
\begin{center}
	--Esta página ha sido dejada intencionalmente en blanco.--
\end{center}
\vfill
\thispagestyle{empty}
\setcounter{page}{2}


%Página de dedicatoria
\chapter*{}
\pagenumbering{Roman} % para comenzar la numeracion de paginas en numeros romanos
\begin{flushright}
	\textit{A mi hija, Ana Magdalena, \\
		que lo mismo toca el clave que el theremín.}
\end{flushright}
\setcounter{page}{3}

%\begin{otherlanguage}{english}
\renewcommand\abstractname{Abstract}
\begin{abstract}
	\setcounter{page}{4}
	\import{capitulos/abstract/}{ingles.tex}
	\thispagestyle{plain}
\end{abstract}
%\end{otherlanguage}
\renewcommand\abstractname{Resumen}
\begin{abstract}
	\setcounter{page}{5}
%	Abstract en español
	\import{capitulos/abstract/}{espanol.tex}
	\thispagestyle{plain}
\end{abstract}


	\setcounter{page}{6}
	\tableofcontents
	\listoffigures % indice de figuras
	\addcontentsline{toc}{chapter}{Índice de figuras}
	\listoftables
	\addcontentsline{toc}{chapter}{Índice de cuadros}
	


	
% Aquí van los diversos capítulos:
	\import{capitulos/Chapter_Introduccion/}{Introduccion.tex}
	\import{capitulos/Chapter_GME_Cuenca/}{Chapter_GME_Cuenca.tex}
	\import{capitulos/Chapter_Otros_trabajos/}{Chapter_Otros_trabajos.tex}
	\import{capitulos/Chapter_Proceso_elaboracion/}{Proceso_trabajo.tex}
	\import{capitulos/Chapter_Descripcion_Aplicacion/}{Chapter_Descripcion_Aplicacion.tex}
	\import{capitulos/Chapter_Futuro/}{Chapter_Futuro.tex}
	\import{capitulos/Chapter_Conclusiones/}{Chapter_Conclusiones.tex}
	\import{capitulos/Chapter_Agradecimientos/}{Chapter_Agradecimientos.tex}

	\appendix
% ANEXOS ************************************
	%\clearpage
	%\addappheadtotoc
	\appendixpage
		\import{anexos/Anexo_instalacion_SynthiEMS/}{Instalacion_SynthiEMS.tex}
		\import{anexos/Anexo_Synthi_100_sheet/}{Synthi_100_sheet.tex}
		\import{anexos/Anexo_Synthi_100_brouchure/}{Synthi_100_brouchure.tex}
	
	

	
% Lista de la bibliografía que será mostrada aunque no esté citada.
	\nocite{Csound_book}
	\nocite{Curtis}
	\nocite{GME}
	\nocite{Lieb}
	\nocite{SC_book}
	\nocite{Supper}
	\nocite{Synthi100_brochure}
	\nocite{Synthi_Australia}
	\nocite{Synthi_users_manual}
	\nocite{the_synthesizer}
	
	\bibliographystyle{apacite}
	
	\bibliographystyle{apacite}
	\bibliography{bibtex/biblio} % database is "biblio.bib" located in a "bibtex" subfolder 
	

\end{document}