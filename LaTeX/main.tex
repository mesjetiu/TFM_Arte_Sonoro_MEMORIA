\documentclass[a4paper,openany,oneside,12pt]{book}

\usepackage[T1]{fontenc}
\usepackage{textcomp}
\usepackage[spanish]{babel} % espanol
\usepackage[utf8]{inputenc} % acentos sin codigo
\renewcommand{\shorthandsspanish}{}

\usepackage{apacite}
\usepackage{titlesec, blindtext, color}
\definecolor{gray75}{gray}{0.75}
\newcommand{\hsp}{\hspace{20pt}}
\titleformat{\chapter}[hang]{\Huge\bfseries}{\thechapter\hsp\textcolor{gray75}{|}\hsp}{0pt}{\Huge\bfseries}
\usepackage{graphicx} % graficos
\usepackage{setspace}
\usepackage{fancyhdr}
\usepackage{eurosym}
\usepackage{verbatim}
\usepackage{xcolor}
%\usepackage{hyperref}
\usepackage{shadow} %Para usar caja con sombra \shabox{}
\usepackage{lettrine}						% Para hacer letras capitales.
\usepackage{rotating}
\usepackage{wasysym} % Para simbolos adicionales
\usepackage{import}

\newcommand{\grad}{\hspace{-2mm}$\phantom{a}^{\circ}$}

\clubpenalty=20000	% para que no haya líneas viudas en el texto.
\widowpenalty=20000

%%% Colores de los links
%\hypersetup{colorlinks=false,
%linkcolor=black,
%urlcolor=blue,
%linkcolor=blue}



%%%%%%%%%%%%%%%%%%%%%%%%%%%%%%%%%%%%%%%%%%%%%%%%%%

\begin{document}
\bibliographystyle{apacite}
%\frontmatter
%	\input{./capitulos/portada}
	\newpage
	\mbox{}
	\thispagestyle{empty} % para que no se numere esta página
%	\pagestyle{fancy}
	


	\tableofcontents
	\listoffigures % indice de figuras
	\addcontentsline{toc}{chapter}{Índice de figuras}
	\listoftables
	\addcontentsline{toc}{chapter}{Índice de cuadros}
	
%\bibliography{bibtex/biblio} % database is "biblio.bib" located in a "bibtex" subfolder 
	



	\import{capitulos/Chapter_Descripcion_Aplicacion/}{Chapter_Descripcion_Aplicacion.tex}
%	esto es una cita~\cite{Freud} y otra~\cite{Einstein}

% Aquí van los diversos capítulos:
	%\input{} % Introducción



\end{document}