%\documentclass[draft, a4paper,openany,oneside,12pt]{report}
\documentclass[a4paper,openany,oneside,12pt]{report}

\usepackage[T1]{fontenc}
\usepackage{textcomp}
\usepackage[spanish]{babel} % espanol
\usepackage[utf8]{inputenc} % acentos sin codigo
\renewcommand{\shorthandsspanish}{}

\usepackage[hidelinks]{hyperref}
\hypersetup{colorlinks=true,
linkcolor=blue,
urlcolor=blue,
linkcolor=blue,
citecolor=red}
\usepackage[apaciteclassic]{apacite}
\usepackage{titlesec, blindtext, color}
\definecolor{gray75}{gray}{0.75}
\newcommand{\hsp}{\hspace{20pt}}
\titleformat{\chapter}[hang]{\Huge\bfseries}{\thechapter\hsp\textcolor{gray75}{|}\hsp}{0pt}{\Huge\bfseries}
\usepackage{graphicx} % graficos
\usepackage{caption}
\usepackage{subcaption}
\usepackage{pdfpages}
\usepackage{setspace}
\usepackage{fancyhdr}
\usepackage{eurosym}
\usepackage{verbatim}
\usepackage{xcolor}
\usepackage{shadow} %Para usar caja con sombra \shabox{}
\usepackage{lettrine}						% Para hacer letras capitales.
\usepackage{rotating}
\usepackage{wasysym} % Para simbolos adicionales
\usepackage{import}
\usepackage{multirow}
\usepackage{svg}
\usepackage{amssymb}
\usepackage{siunitx}
\usepackage{setspace} % Interlineados
\usepackage[toc,page]{appendix}

\sisetup{load-configurations = abbreviations}
\newcommand{\grad}{\hspace{-2mm}$\phantom{a}^{\circ}$}
\clubpenalty=20000	% para que no haya líneas viudas en el texto.
\widowpenalty=20000 % para que no haya huérfanas
% indica a Latex contabilizar el nivel de gerarquía de texto hasta el cuarto nivel;
\setcounter{secnumdepth}{4}
% indica al comando de generación de tabla de contenido que incluya el cuarto nivel:
\setcounter{tocdepth}{4}


%% Title format
\begin{comment}
\titleformat
{\chapter} % command
[display] % shape
{\bfseries\LARGE} % format
{Capítulo \ \thechapter} % label
{0.5ex} % sep
{
	\rule{\textwidth}{1pt}
	\vspace{1ex}
	\centering
} % before-code
[
\vspace{-0.5ex}%
\rule{\textwidth}{0.3pt}
] % after-code
\end{comment}

\renewcommand{\appendixname}{Anexos}
\renewcommand{\appendixtocname}{Anexos}
\renewcommand{\appendixpagename}{Anexos}

\newcommand{\className}{SynthiGME}
\newcommand{\appName}{Synthi \textsc{gme}}
\newcommand{\version}{0.1.0}

%%%%%%%%%%%%%%%%%%%%%%%%%%%%%%%%%%%%%%%%%%%%%%%%%%

\begin{document}
\bibliographystyle{apacite}
 \selectlanguage{spanish}
	\import{./portada/}{portada.tex}
	\newpage
	\mbox{}
	\thispagestyle{empty} % para que no se numere esta página
	\pagestyle{fancy}

\onehalfspace



\begin{abstract}
	\setcounter{page}{2}
	Abstract en español
	\thispagestyle{plain}
\end{abstract}
%\begin{otherlanguage}{english}
\begin{abstract}
	\setcounter{page}{3}
	Abstract in English
	\thispagestyle{plain}
\end{abstract}
%\end{otherlanguage}


	\setcounter{page}{4}
	\tableofcontents
	\listoffigures % indice de figuras
	\addcontentsline{toc}{chapter}{Índice de figuras}
	\listoftables
	\addcontentsline{toc}{chapter}{Índice de cuadros}
	


	
% Aquí van los diversos capítulos:

	\import{capitulos/Chapter_Introduccion/}{Introducción.tex}
	\import{capitulos/Chapter_Objetivos/}{Chapter_Objetivos.tex}
	\import{capitulos/Chapter_Otros_trabajos/}{Chapter_Otros_trabajos.tex}
	\import{capitulos/Chapter_GME_Cuenca/}{Chapter_GME_Cuenca.tex}
	\import{capitulos/Chapter_Descripcion_Aplicacion/}{Chapter_Descripcion_Aplicacion.tex}
	\import{capitulos/Chapter_Futuro/}{Chapter_Futuro.tex}
	\import{capitulos/Chapter_Conclusiones/}{Chapter_Conclusiones.tex}
	
	
	\appendix
% ANEXOS ************************************
	%\clearpage
	%\addappheadtotoc
	\appendixpage
		\import{anexos/Anexo_Synthi_100_sheet/}{Synthi_100_sheet.tex}
		\import{anexos/Anexo_Synthi_100_brouchure/}{Synthi_100_brouchure.tex}
	
	
	CITAS DE PRUEBA: Esto es una cita de ejemplo~\cite{Lieb} y otra~\citeNP{Supper} y otra~\citeA{Curtis} y~\cite{GME}  ~\cite{SC_book} ~\cite{Csound_book}
	
%	\cite[for more details]{GME}
\cite[para más detalles]{Synthi_users_manual}
 Esto es una cita de ejemplo~\cite{Synthi_users_manual} y otra~\citeNP{Synthi_users_manual} y otra~\citeA{Synthi_users_manual} y~\cite{Synthi_users_manual}


	\bibliographystyle{apacite}
	\bibliography{bibtex/biblio} % database is "biblio.bib" located in a "bibtex" subfolder 

\end{document}