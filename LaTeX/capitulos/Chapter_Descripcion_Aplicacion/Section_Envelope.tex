\section{Envelope Shapers}

El <<generador de envolvente>> (conocido por sus nombres en inglés, \textit{Envelope Shaper} o \textit{Envelope Generator}), es uno de los módulos, junto con el de los osciladores, que no suelen faltar en los sintetizadores analógicos. Pero esta omnipresencia no es paralela a su estandarización. Los parámetros que controlan el generador de envolvente dependen de cada fabricante e, incluso, de cada modelo. Por si fuera poco, precisamente una de las grandes diferencias entre el Synthi 100 de Cuenca y el resto de Synthi 100 son los parámetros del módulo \textit{Envelope Shaper}que este permite variar, lo cual da fe de la velocidad a la que el concepto de envolvente se desarrollaba por aquellos años.

\subsection{Las fases de una envolvente típica}

Una <<envolvente>> puede ser descrita como la forma que se da al desarrollo en el tiempo del nivel de un parámetro de una señal, habitualmente la amplitud. De esta forma, una señal continua generada, por ejemplo, con un oscilador, puede adquirir una evolución dinámica, así como un inicio y un final. Puesto que una envolvente puede <<envolver>> a cualquier tipo de señal, sea esta de audio como de voltaje, podemos hacer variar con ella otros parámetros como la frecuencia de un oscilador, la frecuencia de corte de un filtro, el nivel de salida de otra envolvente, etc.

Históricamente, la envolvente surge inspirada por el comportamiento físico de los instrumentos musicales (IMAGEN: FORMA DE ONDA NOTA PIANO), de ahí que el tipo de envolvente más popular hasta nuestros días es el ADSR, siglas de los términos en inglés \textit{attack}, \textit{decay}, \textit{sustain} y \textit{release} (FIGURA: GRÁFICA ENVOLVENTE ADSR):
\begin{description}
	\item[\textit{Attack}] Tiempo transcurrido entre el nivel 0 y el punto álgido (convencionalmente 1). Es el equivalente al \textit{ataque} de un instrumento acústico tipo, en el que los sonidos transitorios pueden producir un rápido pico dinámico.
	\item[\textit{Decay}] Tiempo transcurrido entre el final del ataque, con nivel 1 y el nivel de \textit{sustain}. En instrumentos acústicos, tras alcanzar el nivel máximo en el periodo de ataque, existe un progresivo \textit{decaimiento} dinámico del sonido hasta alcanzar un nivel estable.
	\item[\textit{Sustain}] Nivel en el que la señal puede ser mantenida por tiempo indefinido (entre 0 y 1). A diferencia del resto de parámetros, el \textit{sustain} no es un parámetro temporal, sino de nivel. De hecho, es el único parámetro de nivel de toda la envolvente, ya que el nivel inicial y final siempre son 0 mientras que el nivel alcanzado en el ataque siempre es 1 (entendiendo estos valores como factor de cualquier unidad de medida). En los instrumentos acústicos, este nivel es mantenido a voluntad por el instrumentista. En los sintetizadores (analógicos o digitales) se utiliza el control de \textit{gate} para mantener la señal en \textit{sustain}.
	\item[\textit{Release}] Tiempo transcurrido entre el final del mantenimiento de la señal (\textit{sustain}), y su extinción. 
\end{description} 

\subsection{Control de la envolvente: \textit{Gate} y \textit{trigger}}

Como se ha indicado, \textit{attack}, \textit{decay} y \textit{release} son duraciones, con lo que su inicio y fin están determinados por los valores elegidos por el usuario (o por el control desde otro módulo), pero ¿cómo sabe el sintetizador cuándo ha de terminar \textit{sustain} y comenzar el periodo de \textit{release}? Y otra pregunta muy relacionada, ¿cómo se da comienzo al periodo de \textit{attack}? La analogía de un instrumento de tecla, como el órgano, puede darnos una respuesta muy intuitiva: Cuando el organista baja la tecla se pone en marcha el ciclo completo de la nota musical con su propia envolvente dinámica. En ese preciso instante comienza un periodo de crecimento sonoro producido por el propio aire entrando a los tubos como por las sucesivas reflexiones del sonido recién comenzado en las paredes de la sala. Tanto el \textit{ataque} como el \textit{decaimiento} tienen lugar con un tiempo definido para cada tubo, con controlable por el intérprete. Una vez estabilizado el sonido en un nivel determinado (\textit{sustain}) el sonido continuará este nivel tanto tiempo como quiera el intérprete. Solo cuando levante la tecla a su posición inicial será cuando se de comienzo a la fase de \textit{release}, en la que el sonido, dependiendo de las características de la sala, requerirá de un periodo de tiempo mayor o menor hasta dejar de ser escuchado.

La tecla \textit{bajada} y \textit{subida} tiene su análogo dentro del mundo de los sintetizadores en el concepto de \textit{gate} (\textit{<<puerta>>}), la cual se puede encontrar, del mismo modo, en dos estados: \textit{abierta} (valor mayor que 0) y \textit{cerrada} (valor de 0 o inferior). El valor de este \textit{gate} no es otra cosa que un nivel de voltaje en el caso de los sintetizadores analógicos o el valor digital de 1 o 0 en el caso de un dispositivo de \textit{software}. En ambos casos, \textit{gate} puede ser considerada otra señal, cuya función es la de <<dirigir>> las etapas de la envolvente. 

Si la señal \textit{gate} viene delimitada por su inicio y su fin ---puntos con los que se ponen en marcha los diferentes periodos de una envolvente--- el \textit{trigger} es cualquier señal en cuando pasa de un valor de 0 o negativo hasta un valor mayor que 0. Es el inicio de dicha señal (momento en el que pasa a ser mayor que 0) el que puede ser usado como \textit{trigger}. Para el desarrollo de una envolvente en la que no existe \textit{sustain} bastaría una señal \textit{trigger} para dar el <<pistoletazo de salida>> y recorrer sucesivamente todas sus etapas de principio a fin, con una duración total suma de todas ellas: \textit{attack}, \textit{decay} y \textit{release}. 


FIGURA: GRÁFICA DE GATE Y TRIGGER

\subsection{Otros tipos de envolventes}

No hay nada que nos limite a tres duraciones y transiciones (\textit{attack}, \textit{decay} y \textit{release}) y a un nivel variable controlado con \textit{gate}, \textit{sustain}. De hecho, ADSR no es más que un caso especial de toda una infinidad de tipos de envolvente imaginables, especialmente en el mundo digital, donde podemos diseñar tantos periodos de transición como se quiera, asociados también a valores arbitrarios. Sin embargo, en los sintetizadores analógicos no hay más remedio que delimitar el número de parámetros de la envolvente, ya que cada uno de ellos será un dial. Por ello, podemos encontrar cierta variedad de envolventes que, en definitiva, podrían ser explicadas por configuraciones ADSR con los valores apropiados. 

TABLA: diferentes envolventes explicadas por ADSR

\subsection{El generador de envolventes del Synthi 100}

Diferencias entre la envolvente del Synthi 100 original y el de Cuenca.

Comportamiento anómalo y los compromisos adoptados en el emulador.

(Crear tabla con las diferentes envolventes, las del Synthi 100, las del Synthi 100 de Cuenca y el emulador.)