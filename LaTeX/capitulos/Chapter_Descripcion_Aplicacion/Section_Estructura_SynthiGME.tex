 \section[Estructura de \appName]{Estructura del \textit{quark} \appName \sectionmark{Estructura de \appName}}
\sectionmark{Estructura de \appName}

A continuación se muestra el árbol de directorios en los que se organiza el \textit{Quark} de \appName. En la representación no se muestran los archivos concretos que albergan las definiciones de las clases, a excepción del de la clase principal \texttt{SynthiGME} y los que están alojados en el directorio raíz. 


\begin{figure}[H]
	\singlespace
	\dirtree{%
		.1 classes\DTcomment{ 
			\begin{minipage}[t]{8cm}
				Todas las clases de SuperCollider de \appName.
		\end{minipage}}.
		.2 GUI\DTcomment{ 
			\begin{minipage}[t]{8cm}
				Todas las clases, gráficos e imágenes de la interfaz gráfica de usuario.
		\end{minipage}}.
		.3 images.
		.4 panels.
		.4 widgets.
		.2 modules\DTcomment{ 
			\begin{minipage}[t]{8cm}
				Las clases que representan a los módulos del Synthi 100.
		\end{minipage}}.
		.2 SGME\_Settings\DTcomment{ 
			\begin{minipage}[t]{8cm}
				Archivos de configuración (límites, rangos y otros niveles modificables).
		\end{minipage}}.
		.2 SynthiGME.sc\DTcomment{ 
			\begin{minipage}[t]{8cm}
				Clase principal.
		\end{minipage}}.
		.1 HelpSource\DTcomment{ 
			\begin{minipage}[t]{8cm}
				Archivos de ayuda y documentación.
		\end{minipage}}.
		.1 Installation\DTcomment{ 
			\begin{minipage}[t]{8cm}
				Scripts de ayuda y ejemplos.
		\end{minipage}}.
		.1 Ladish studio.
		.1 TouchOSC.
		.1 COPYING.
		.1 README.md.
		.1 SynthiGME.quark\DTcomment{ 
			\begin{minipage}[t]{8cm}
				Archivo de información del \textit{Quark} (para SuperCollider).
		\end{minipage}}.
	}
\caption[Árbol de directorios de \appName]{Árbol de directorios de \appName.}
	\label{fig:arbol_directorios}
\end{figure}

