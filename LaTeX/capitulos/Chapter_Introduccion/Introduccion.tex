\chapter[Introducción]{Introducción}
\chaptermark{Introducción}

iDEAS POR TERMINAR DE REDACTAR

Utopía del sonido desencarnado (ideas para redactar)

La publicidad de la época resalta la capacidad ilimitada de creación sonora de los sintetizadores.

Pero igual que reconocemos una pieza creada con un instrumento clásico, podemos reconocer los sonidos de un sintetizador. A pesar de ser ilimitadas las posibilidades, existen ciertos patrones reconocibles: sonidos puros, frecuencias moduladas, glissandos, etc. 

De hecho, no es más ilimitado el abanico de sonidos que se pueden arrancar de un sintetizador respecto al que se puede arrancar de una orquesta. Cada uno tiene sus propios patrones reconocibles. 

La llegada de las computadoras al mundo sonoro prometía una definitiva desencarnación del sonido respecto a las limitaciones. Este optimismo aún se puede encontrar. No hace muchos años, un profesor mío de música electroacústica me mostraba el potencial de Csound diciendo: <<Componer con música electrónica es tener ante sí los infinitos sonidos posibles>>. Este optimismo me cautivó y me pareció creíble. Sin embargo, poco a poco me fui dando cuenta de que todo lo que podía hacer con el ordenador, era tan reconocible como <<música de ordenador>> como lo era la música creada con el sintetizador o la creada con la orquesta sinfónica. 

Lo que un artista puede crear está determinado por lo que el instrumento puede hacer potencialmente, por una parte, y de su gesto, por la otra. Herramienta y gesto son inseparables en la creación de cualquier arte, y el sonoro no es una excepción. Pero es muy tentador pensar que las herramientas más modernas son las que definitivamente están desencarnadas de todo gesto y que son potencialmente ilimitadas y que engloban a las ya obsoletas. Nada más lejos de la realidad. Ningún sintetizador analógico consiguió imitar a la perfección un instrumento musical clásico. Este problema se puede apreciar muy claramente en la evolución de los órganos electrónicos. En mi experiencia como organista, he podido tocar todo tipo de instrumentos electrónicos que prometen un sonido idéntico al de un órgano de tubos, y hasta la fecha no he encontrado ninguna solución que sea totalmente convincente. De hecho, solo los órganos que usan alguna suerte de sampleado en sus sonidos son los que más se aproximan a un sonido salido de tubos de órgano.

Lo mismo hemos de concluir con la síntesis digital. Aquella infinitud que se me prometía en el estudio de Csound, representa más un cambio de paradigma que la panacea del sonido. Cambia el gesto, que ya no es un arco de un violín o una perilla de un sintetizador. Ahora puede ser el tecleo de código, el movimiento del ratón o de cualquier dispositivo que se nos ocurra. Pero las posibilidades siempre vendrá determinadas por dicho gesto y por lo que el software nos permita.

El corolario de este argumento es que tan burda es la imitación de un órgano o un violín por un sintetizador analógico como lo es la de este por medios computacionales. Quizás el resultado sonoro en este segundo caso sea más convincente que en el primero, pero no hay que olvidar que el gesto ha cambiado indefectiblemente. No es lo mismo mover el ratón que estirar el brazo para girar una perilla al tiempo que se conecta con la otra mano la salida de un modulo. Entonces, ¿por qué emular un sintetizador analógico? Creo que la principal razón es la de devolverle la vida que la obsolescencia le ha robado. Un emulador no sustituye nunca a lo emulado, pero lo da a conocer, y diseñarlo es una forma muy gratificante de entender por qué es cómo es, por qué tiene unas características y no otras, por qué cuenta con unos controles y no con otros que esperaríamos, etc. 

La primera vez que oí hablar del Synthi 100 del Gabinete de Música Electroacústica de mi ciudad fue en una grata conversación con el compositor Julio Sanz Vázquez en enero de 2019. No solo me hizo un magistral resumen de la historia del gabinete y de las personas más importantes con él relacionadas, sino que me habló del elemento estrella del estudio, su flamante sintetizador. En menos de un año tras la conversación ya me había propuesto realizar una emulación digital del mismo, que coincidiría precisamente con los trabajos de restauración que la Universidad de Cuenca, que alberga hoy el gabinete, comenzaba en ese momento. En este contexto mi trabajo adquiría para mí un gran interés, ya que podría servir como un elemento divulgador y pedagógico en torno a la recuperación del Synthi. 

Durante los meses en los que me he dedicado a la investigación e implementación de este emulador, he contado con la inestimable ayuda, apoyo y ánimos tanto de Julio Sanz como de Sylvia Molina, quienes me han abierto de par en par las puertas del GME y me han permitido llegar hasta las perillas del Synthi 100. Por desgracia, la crisis nacional de los últimos meses ha paralizado todo este contacto fundamental con el sintetizador. No obstante, ha habido suficiente tiempo para hacer un exhaustivo reportaje fotográfico del Synthi 100, probar ciertos modulos y preguntar muchas dudas ante la ausencia de manuales detallados de la época.


(Hablar en algún sitio de la vacía discusion entre qué suena mejor, lo analógico o lo digital)

**********
¿Qué no es Synthi GME?

**********
Insertar aquí los objetivos


***************************************
explicar el nombre de la aplicación



\subsection{sobre el nombre \appName}

Hay una asentada tradición en el software libre de que ciertos nombres de programas sean recursivos. KDE, GNU