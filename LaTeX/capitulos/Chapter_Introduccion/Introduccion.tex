\chapter[Introducción]{Introducción}
\chaptermark{Introducción}
\pagenumbering{arabic}

\section{El sonido, la herramienta y el gesto}
\sectionmark{El sonido, la herramienta\dots}


No es infrecuente encontrar folletos publicitarios de sintetizadores de las décadas de los 70 y 80 en los que se resalta la capacidad ilimitada de creación sonora de estos artefactos. En la introducción del manual de uso de Synthi AKS \cite{SynthiAKS_brochure}, por ejemplo, se alaba especialmente el hecho de que el sintetizador, a diferencia de otros instrumentos, no tiene un sonido característico. Esta idea de universalidad del sonido sintetizado también se puede encontrar en revistas especializadas. Así, Pérez-Arroyo describía el funcionamiento del sintetizador como \textit{<<capaz de producir cualquier tipo de sonido existente o no, por medios electrónicos>>} \cite{ritmo_542}. 

Pero igual que reconocemos un sonido producido con un instrumento clásico, no tengo duda de que también podemos reconocer los sonidos procedentes de un sintetizador. A pesar de ser ilimitadas las posibilidades y de no tener un timbre característico al modo de un instrumento clásico, existen ciertos indicios reconocibles: sonidos puros, frecuencias moduladas, formas <<matemáticas>> de los \textit{glissandos}, las reverberaciones artificiales, etc., incluso cuando sus combinaciones llegan a gran complejidad. De hecho, me aventuro a decir que no es más ilimitado el abanico de sonidos que se pueden arrancar de un sintetizador respecto al que se puede extraer de una orquesta clásica. Esta discusión me trae a la mente la cuestión matemática de que aunque el conjunto de los números impares es mayor que el de los números primos, en ambos casos se trata de conjuntos infinitos. Y el hecho de que uno sea más grande no nos permite suponer que uno sea superconjunto de otro. De hecho no lo son. Algo análogo ocurre, a mi modo de ver, con los sonidos sintetizados respecto a los acústicos. Cada uno tiene sus propios patrones reconocibles, sus señas de identidad sonora. En ambos casos las posibilidades son infinitas, puede que mayores los del sonido sintetizado electrónicamente, pero ello no le confiere el derecho de erigirse en ningún caso en una generalización del sonido de origen físico. Desde luego, no son necesarios hoy argumentos para mostrar lo que es un hecho. Aquel optimismo respondía más a un cambio de paradigma que a una realidad comprobada.

La llegada de las computadoras al mundo sonoro parecía prometer una definitiva desencarnación del sonido respecto a las limitaciones. Este optimismo no se circunscribe a los inicios de los ordenadores ---que bañaba a otros muchos campos más, como la inteligencia artificial---, sino que sigue siendo común en nuestros días. No hace muchos años, un profesor mío de música electroacústica me mostraba el potencial de CSound, como si en ese archivo en blanco antes de escribir código alguno, estuvieran en potencia ante nosotros <<los infinitos sonidos posibles>>. Esta idea de infinitud me cautivó y confieso que me pareció creíble y sugerente durante mucho tiempo. Era algo así como estar ante la \textit{Biblioteca de Babel}\footnote{\textit{La Bioblioteca de Babel} es un afamado libro de José Luis Borges, en el que plantea la idea de infinitud unida a una biblioteca imaginaria en la que existen simultáneamente todos los libros que es posible escribir con todas las combinaciones del alfabeto y los signos de puntuación.} del sonido. La lectura de este libro siempre ha hecho que mi imaginación vuele hacia otra biblioteca imaginaria, la del sonido digital. No hay más que delimitar el formato de los archivos de esta biblioteca y dejar que existan todas las combinaciones posibles de unos y ceros en un disco duro ideal. ¡Esto sí que sería tener posibilidades sonoras delante! El problema es que dejar abiertas todas las posibilidades arroja principalmente <<ruido>>. Revisar al azar un libro de la biblioteca de Babel no es muy diferente que pedir a un ordenador que nos entregue un libro con letras al azar. No importa las veces que hagamos click para ver un libro nuevo, con toda seguridad encontraremos letras sin sentido. Lo mismo ocurre cuando generamos un segundo de audio con bits al azar. Cualquiera que se dedique al audio digital sabe que el resultado será <<ruido blanco>>, y en caso negativo, se debe a algún fallo en el algoritmo. Un sintetizador digital no tiene ante sí todas las posibilidades ni mucho menos. Ni siquiera el ruido que genera es absolutamente azaroso. Todo lo que puede hacer un ordenador a nivel sonoro, es tan reconocible, como lo es el sonido creado con un sintetizador o el creado con una orquesta sinfónica. Solo nuestra particular Biblioteca de Babel sonora contiene todos los sonidos posibles, pero, precisamente por ser tantos, nos es infinitamente más fácil <<crearlo>> que buscarlo.

Y aquí viene al auxilio otro elemento que considero esencial en toda creación: el <<gesto>>. Lo que un artista puede crear está determinado fuertemente por lo que el instrumento puede hacer potencialmente, por una parte, y del gesto humano, por  otra. <<Herramienta>> y <<gesto>> son inseparables en la creación de cualquier arte, y el sonoro ---en el sentido más amplio posible: sonido, música, ruido\dots--- no es una excepción. Es, sin duda, muy tentador pensar que las herramientas más modernas son las que definitivamente están desencarnadas de todo gesto, que son potencialmente ilimitadas y que engloban a las ya obsoletas. Pero un producto sonoro (producido con intencionalidad humana) sin gesto no existe; siempre lleva consigo la impronta del gesto productor. Ningún sintetizador analógico consiguió imitar a la perfección un instrumento musical clásico. Este hecho se puede apreciar muy claramente en la evolución de los órganos electrónicos. En mi experiencia como organista, he podido tocar todo tipo de instrumentos electrónicos que prometen un sonido idéntico al de un órgano de tubos, y hasta la fecha no he encontrado ninguna solución que sea totalmente convincente. Finalmente, solo los órganos que usan alguna suerte de sampleado en sus sonidos son los que más se aproximan a los  salidos de tubos de órgano. No quiero entrar en una discusión sin fin de si una muestra sonora es un sonido digital. Desde luego que bajo algún aspecto sí que lo es, pero descarto este tipo de sonido por ser una copia obvia de otro sonido generado físicamente. De hecho, basta procesar estos samples, aplicarles ciertas técnicas propias de la música concreta o electroacústica para que, de nuevo, nuestros oídos reconozcan que el producto ha salido de un ordenador o que ha pasado por procesos de cinta magnética. Nuestra mente oye el gesto y lo entiende, porque el gesto es humano. Oímos un violín y parece que vemos las arcadas, oímos un cantante y el fraseo nos recuerda el gesto de respirar\dots 

A veces el gesto no es obvio. Oimos sonidos y no los podemos asociar a su fuente, ni comprender la intención del que los genera, porque no <<vemos>> el gesto, nos es imposible identificarlo. No me estoy refiriendo con esto al sonido <<desencarnado>> del <<objeto sonoro>> de Pierre Schaeffer, que le sirve de ladrillo para crear, sino, más bien, al producto sonoro ya terminado. No cabe duda de que la música de índole electrónica o digital puede desconcertar o sorprender al oyente ya que no se le comunica siempre de forma obvia el gesto que la ha producido. Pero el gesto ha estado ahí: la escritura de un código, el deslizamiento de unos \textit{sliders}, el giro de una perilla. De hecho, basta escuchar un puñado de obras que han utilizado la misma herramienta para que nuestra mente aprenda a reconocerla.

Las posibilidades siempre vendrán determinadas por dicho gesto y por lo que la herramienta ---sea software, sea sintetizador--- nos permita. Aquella infinitud que se me prometía en el estudio de Csound, representa más un cambio de paradigma que la panacea de la creación sonora. Solo cambia el gesto, que ya no es el de un arco de un violín o una perilla de un sintetizador. Ahora puede ser el tecleo de código, el movimiento del ratón o de cualquier dispositivo que se nos ocurra. 

El corolario de toda argumentación es que tan burda es la imitación de un órgano o un violín por un sintetizador analógico como lo es la de este por medios computacionales. Quizás el resultado sonoro en este segundo caso sea más convincente que en el primero, pero no hay que olvidar que el gesto ha cambiado indefectiblemente. No es lo mismo mover el ratón que estirar el brazo para girar una perilla al tiempo que se conecta con la otra mano la salida de un modulo. Es claro que no podemos hacer lo mismo con un sintetizador que con su emulación en una pantalla, pero aun así siempre han existido emulaciones: sintetizadores que buscan imitar sonidos de instrumentos acústicos, sintetizadores digitales que emulan a sintetizadores analógicos, sintetizadores digitales que emulan a instrumentos acústicos. Unas veces son un curioso juguete, otras una forma pretendida de <<sustituir>> a lo emulado. En todo caso los veo como una ocasión para descubrir la originalidad de cada técnica y cada fuente, su esencia y su valor artístico. Vuelvo a mi oficio de organista; me encantan los emuladores de órganos, los uso y me interesa su funcionamiento. Pero nunca he sentido que ya no fuese necesario mantener caros y delicados instrumentos acústicos. Muy al contrario, la emulación me remite a lo emulado y cuanto más se consigue el propósito de imitación, más valoro lo imitado y su originalidad.

Esta es la razón de fondo de este trabajo. Emular un sintetizador analógico supone todo un reto multidisciplinar: acústica, psicoacústica, síntesis analógica, electrónica, computación, sonido digital, historia, interfaz y su gesto, etc. En el proceso aparecen indefectiblemente efectos triviales para un sintetizador analógico que son muy complejos para uno digital, y viceversa, lo cual redunda en la idea de que cada disciplina es independiente y que una no engloba a la otra. 

Difícilmente voy a reinventar la rueda diseñando un emulador --ni lo intento--, pero si pongo el conocimiento del proceso, su código y la posibilidad de uso, en manos de la comunidad, quizás el trabajo adquiera entonces sentido. Confío en que este trabajo pueda despertar el interés de otras personas, como lo hizo conmigo, por la síntesis analógica y los instrumentos <<históricos>> que en ocasiones se han visto como chatarra inservible. Diseñar un dispositivo como este, objeto del trabajo, es una forma muy gratificante de entender por qué un sintetizador es cómo es, por qué tiene unas características y no otras, por qué fue diseñado con unos controles y no con otros que quizás esperaríamos, cuáles son sus posibilidades; en definitiva, el porqué de la herramienta y sus gestos inherentes.

\section{Esta memoria y \appName}

El presente texto es una memoria sobre el proceso de creación de un emulador digital del sintetizador modular EMS Synthi 100 del Gabinete de Música Electroacústica de Cuenca. La memoria consta principalmete de una exhaustiva descripción de los módulos implementados, la decisiones tomadas en cada caso, así como la explicación de su funcionamiento para el usuario.

El dispositivo de software se ha programado para PC, en el lenguaje de SuperCollider. Se ha diseñado con un fin principalmente pedagógico y por ello ha sido licenciado bajo los términos de GNU GPLv3 \citeyear{gpl}. 

La aplicación se llama <<\appName>>, nombre recursivo siguiendo la tradición de muchos programas de software libre (\textit{GNU}, \textit{KDE}, etc.). En él, <<GME>> son las siglas de <<GME Modular Emulator>>, donde <<GME>> vuelve a significar de nuevo <<GME Modular Emulator>>, y así \textit{ad infinitum}, haciendo un guiño al Gabinete de Música Electroacústica, que es lo que significa \textit{GME} en su última iteración\dots


\section{Objetivos}
\label{sec:objetivos}

Por claridad, he dividido los objetivos del trabajo en dos secciones. En la primera, los \textit{objetivos generales} describen las metas y motivaciones más globales y remotas que hay tras el proyecto. La segunda seccion, los \textit{objetivos específicos}, enumera los objetivos más inmediatos y concretos en la elaboración de este proyecto.


\subsection{Objetivos generales}


\begin{enumerate}
	\item Promover el conocimiento de la síntesis sonora analógica, como medio para comprender las nuevas técnicas digitales.
	\item Revalorizar y dar a conocer el patrimonio del arte sonoro español, tangible e intangible, en muchos casos relegado a su almacenamiento, con vistas a fomentar su restauración y puesta a punto y publicación.
	\item Fomentar el uso del software libre tanto en el ámbito técnico como artístico, como medio idóneo de transmisión y creación de conocimiento.
	\item Contribuir a la proliferación de herramientas pedagógicas libres y colaborativas en el ámbito universitario y profesional.
	\item Poner a disposición de la comunidad educativa en general, y del GME de Cuenca en particular, una herramienta didáctica y de promoción del patrimonio instrumental del arte sonoro.
	\item Alentar la formación multidisciplinar y multicompetencial en el ámbito de la creación artística.
\end{enumerate}





\subsection{Objetivos específicos}

En cuanto a la creación de la aplicación informática, se destacan los siguientes objetivos:

\begin{enumerate}
	
	\item Diseñar y crear un programa informático que emule la interfaz y el funcionamiento del sintetizador EMS Synthi 100 del Gabinete de Música Electroacústica de Cuenca.
	\item Poner el foco en el uso pedagógico y artístico del dispositivo de software.
	\item Conferirle una arquitectura escalable con vistas a constantes mejoras y ampliaciones.
	\item Publicarse como software libre en los medios habituales de este momento.
	\item Implementar una interfaz de usuario abierta al uso de otros dispositivos tanto de software como de hardware.
	
\end{enumerate}

La presente memoria sobre el diseño y programación de esta herramienta de software tiene por objetivos:

\begin{enumerate}
	\setcounter{enumi}{5}
	\item Ofrecer una descripción del funcionamiento de los principales módulos del sintetizador, atendiendo tanto a las similitudes como a las diferencias con versiones anteriores del mismo modelo.
	\item Servir de manual de la aplicación informática, sin perjuicio de la creación de los archivos de ayuda propios de la misma.
	\item Recoger información y contribuir a la escasa documentación sobre el funcionamiento del EMS Synthi 100 del GME.
\end{enumerate}



\begin{enumerate}
	\setcounter{enumi}{5}
	\item Poner a disposición de la comunidad educativa en general, y del GME de Cuenca en particular, una herramienta didáctica y de promoción del patrimonio instrumental del arte sonoro.
\end{enumerate}



\section[Mi relación con el Synthi 100\dots]{Mi relación con el Synthi 100 del GME de Cuenca \sectionmark{Mi relación con el Synthi 100\dots}}
\sectionmark{Mi relación con el Synthi 100\dots}

La primera vez que oí hablar del Synthi 100 del Gabinete de Música Electroacústica de mi ciudad fue en una grata conversación con el compositor Julio Sanz Vázquez en enero de 2019. No solo me hizo un magistral resumen de la historia del gabinete y de las personas más importantes con él relacionadas, sino que me habló del elemento estrella del estudio, su flamante sintetizador. En menos de un año tras la conversación ya pude comenzar a realizar una emulación digital del mismo, momento que coincidiría precisamente con los trabajos de restauración que la Universidad de Castilla--La Mancha en Cuenca, que alberga hoy el gabinete, comenzaba en ese momento. En este contexto mi trabajo adquiría para mí un gran interés, ya que podría servir como un elemento divulgador y pedagógico en torno a la recuperación del Synthi. 

Durante los meses en los que me he dedicado a la investigación e implementación de este emulador, he contado con el apoyo de la universidad, gracias tanto a Sylvia Molina como a Julio Sanz, quienes me han abierto de par en par las puertas del GME y me han permitido llegar hasta las perillas del Synthi 100. Por desgracia, la crisis que se está viviendo en España durante los últimos meses ha paralizado todo este contacto fundamental con el sintetizador. No obstante, ha habido suficiente tiempo como para hacer un exhaustivo reportaje fotográfico del Synthi 100, probar y experimentar con ciertos módulos, y preguntar muchas dudas ante la ausencia de manuales detallados de la época.


