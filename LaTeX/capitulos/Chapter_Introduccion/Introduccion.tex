\chapter[Introducción]{Introducción}
\chaptermark{Introducción}
\pagenumbering{arabic}

\section{El sonido, la herramienta y el gesto}
\sectionmark{El sonido, la herramienta\dots}


No es infrecuente encontrar folletos publicitarios de sintetizadores de los 70 y 80 en los que se resalta la capacidad ilimitada de creación sonora de estos artefactos. En la introducción del manual de uso de Synthi AKS \cite{SynthiAKS_brochure} alaba especialmente el hecho de que el sintetizador, a diferencia de otros instrumentos, no tiene un sonido característico. Esta idea de universalidad del sonido sintetizado también se puede encontrar en revistas especializadas. Así, Pérez-Arroyo describía el funcionamiento del sintetizador como \textit{<<capaz de producir cualquier tipo de sonido existente o no, por medios electrónicos>>}\cite{ritmo_542}. 

Pero igual que reconocemos una pieza creada con un instrumento clásico, podemos reconocer los sonidos procedentes de un sintetizador. A pesar de ser ilimitadas las posibilidades, existen ciertos indicios reconocibles: sonidos puros, frecuencias moduladas, glissandos, etc., incluso cuando sus combinaciones llegan a gran complejidad. De hecho, me aventuro a decir que no es más ilimitado el abanico de sonidos que se pueden arrancar de un sintetizador respecto al que se puede extraer de una orquesta clásica. Cada uno tiene sus propios patrones reconocibles, sus señas de identidad sonora. 

La llegada de las computadoras al mundo sonoro prometía una definitiva desencarnación del sonido respecto a las limitaciones. Este optimismo no se circunscribe a los inicios de los ordenadores, sino que sigue siendo común en nuestros días. No hace muchos años, un profesor mío de música electroacústica me mostraba el potencial de CSound como tener ante sí los infinitos sonidos posibles. Esta idea de infinitud me cautivó y me pareció creíble. Sin embargo, poco a poco me fui dando cuenta de que todo lo que podía hacer con el ordenador, era tan reconocible como <<música de ordenador>> como lo era la música creada con el sintetizador o la creada con la orquesta sinfónica como tales. 

Lo que un artista puede crear está determinado fuertemente por lo que el instrumento puede hacer potencialmente, por una parte, y de su gesto, por  otra. <<Herramienta>> y <<gesto>> son inseparables en la creación de cualquier arte, y el sonoro no es una excepción. Es, sin duda, muy tentador pensar que las herramientas más modernas son las que definitivamente están desencarnadas de todo gesto y que son potencialmente ilimitadas y que engloban a las ya obsoletas. Pero nada más lejos de la realidad. Ningún sintetizador analógico consiguió imitar a la perfección un instrumento musical clásico. Este problema se puede apreciar muy claramente en la evolución de los órganos electrónicos. En mi experiencia como organista, he podido tocar todo tipo de instrumentos electrónicos que prometen un sonido idéntico al de un órgano de tubos, y hasta la fecha no he encontrado ninguna solución que sea totalmente convincente. Finalmente, solo los órganos que usan alguna suerte de sampleado en sus sonidos son los que más se aproximan a un sonido salido de tubos de órgano, pero no se trata en ningún caso de sonidos sintetizados.

Lo mismo hemos de concluir en el caso la emulación digital de dispositivos analógicos. Aquella infinitud que se me prometía en el estudio de Csound, representa más un cambio de paradigma que la panacea del sonido. Cambia el gesto, que ya no es el de un arco de un violín o una perilla de un sintetizador. Ahora puede ser el tecleo de código, el movimiento del ratón o de cualquier dispositivo que se nos ocurra. Pero las posibilidades siempre vendrán determinadas por dicho gesto y por lo que el software, la herramienta, nos permita.

El corolario de este argumento es que tan burda es la imitación de un órgano o un violín por un sintetizador analógico como lo es la de este por medios computacionales. Quizás el resultado sonoro en este segundo caso sea más convincente que en el primero, pero no hay que olvidar que el gesto ha cambiado indefectiblemente. No es lo mismo mover el ratón que estirar el brazo para girar una perilla al tiempo que se conecta con la otra mano la salida de un modulo. Entonces, ¿por qué emular un sintetizador analógico? Las razones pueden ser tantas como los usuarios de estos productos demanden, pero en el caso de este trabajo la principal razón es la de devolverle la vida que la obsolescencia le ha robado a un sintetizador histórico. Cierto que un emulador no sustituye nunca a lo emulado, pero lo da a conocer y pone de manifiesto su valor con todo el respeto, y diseñarlo es una forma muy gratificante de entender por qué es cómo es, por qué tiene unas características y no otras, por qué cuenta con unos controles y no con otros que esperaríamos, etc. 

\section{Sobre este trabajo}

El presente texto es una memoria sobre el proceso de creación de un emulador digital del sintetizador modular EMS Synthi 100 del Gabinete de Música Electroacústica de Cuenca. La memoria consta principalmete de una exhaustiva descripción de los módulos implementados, la decisiones tomadas en cada caso, así como la explicación de su funcionamiento para el usuario.

El dispositivo de software se ha programado para PC, en el lenguaje de SuperCollider. Se ha diseñado con un fin principalmente pedagógico y por ello ha sido licenciado bajo los términos de GNU GPLv2. 

La aplicación se llama \appName, nombre recursivo siguiendo la tradición de muchos programas de software libre (\textit{GNU}, \textit{KDE}, etc.). En él, <<GME>> son las siglas de <<GME Modular Emulator>>, donde <<GME>> vuelve a significar de nuevo <<GME Modular Emulator>> \textit{ad infinitum}, haciendo un guiño al Gabinete de Música Electroacústica, que es lo que significa \textit{GME} en su última iteración\dots


\section{Objetivos}
\label{sec:objetivos}

Por claridad, he dividido los objetivos del trabajo en dos secciones, en la primera, los \textit{objetivos generales} se centran en las metas y motivaciones más globales y remotas tras el proyecto. La segunda seccion, de los \textit{objetivos particulares}, enumera los objetivos más inmediatos y concretos en la elaboración de este proyecto.


\subsection{Objetivos generales}


\begin{enumerate}
	\item Promover el conocimiento de la síntesis sonora analógica, como medio para comprender las nuevas técnicas digitales.
	\item Revalorizar y dar a conocer el patrimonio del arte sonoro español, tangible e intangible, en muchos casos relegado a su almacenamiento, con vistas a fomentar su restauración y puesta a punto y publicación.
	\item Fomentar el uso del software libre tanto en el ámbito técnico como artístico, como medio idóneo de transmisión y creación de conocimiento.
	\item Contribuir a la proliferación de herramientas pedagógicas libres y colaborativas en el ámbito universitario y profesional.
	\item Poner a disposición de la comunidad educativa en general, y del GME de Cuenca en particular, una herramienta didáctica y de promoción del patrimonio instrumental del arte sonoro.
	\item Alentar la formación multidisciplinar y multicompetencial en el ámbito de la creación artística.
\end{enumerate}





\subsection{Objetivos específicos}

Dividimos los objetivos específicos en diferentes bloques. En cuanto a la creación de la aplicación informática, se destacan los siguientes:

\begin{enumerate}
	
	\item Diseñar y crear un programa informático que emule la interfaz y el funcionamiento del sintetizador EMS Synthi 100 del Gabinete de Música Electroacústica de Cuenca.
	\item Poner el foco en el uso pedagógico y artístico.
	\item Conferirle una arquitectura escalable con vistas a constantes mejoras y ampliaciones.
	\item Publicarse como software libre en los medios habituales de este momento.
	\item Implementar una interfaz de usuario abierta al uso de otros dispositivos tanto de software como de hardware.
	
\end{enumerate}

La presente memoria sobre diseño y programación de esta herramienta de software tiene por objetivos:

\begin{enumerate}
	\setcounter{enumi}{5}
	\item Ofrecer una descripción del funcionamiento de los principales módulos del sintetizador, atendiendo tanto a las similitudes como a las diferencias con versiones anteriores del mismo modelo.
	\item Servir de manual de la aplicación informática, sin perjuicio de la creación de los archivos de ayuda propios de la misma.
	\item Recoger información y contribuir a la escasa documentación sobre el funcionamiento del EMS Synthi 100 del GME.
\end{enumerate}



\begin{enumerate}
	\setcounter{enumi}{5}
	\item Poner a disposición de la comunidad educativa en general, y del GME de Cuenca en particular, una herramienta didáctica y de promoción del patrimonio instrumental del arte sonoro.
\end{enumerate}



\section[Mi relación con el Synthi\dots]{Mi relación con el Synthi 100 del GME de Cuenca \sectionmark{Mi relación con el Synthi\dots}}
\sectionmark{Mi relación con el Synthi\dots}

La primera vez que oí hablar del Synthi 100 del Gabinete de Música Electroacústica de mi ciudad fue en una grata conversación con el compositor Julio Sanz Vázquez en enero de 2019. No solo me hizo un magistral resumen de la historia del gabinete y de las personas más importantes con él relacionadas, sino que me habló del elemento estrella del estudio, su flamante sintetizador. En menos de un año tras la conversación ya me había propuesto realizar una emulación digital del mismo, que coincidiría precisamente con los trabajos de restauración que la Universidad de Cuenca, que alberga hoy el gabinete, comenzaba en ese momento. En este contexto mi trabajo adquiría para mí un gran interés, ya que podría servir como un elemento divulgador y pedagógico en torno a la recuperación del Synthi. 

Durante los meses en los que me he dedicado a la investigación e implementación de este emulador, he contado con la inestimable ayuda, apoyo y ánimos tanto de Julio Sanz como de Sylvia Molina, quienes me han abierto de par en par las puertas del GME y me han permitido llegar hasta las perillas del Synthi 100. Por desgracia, la crisis que se está viviendo en España los últimos meses ha paralizado todo este contacto fundamental con el sintetizador. No obstante, ha habido suficiente tiempo para hacer un exhaustivo reportaje fotográfico del Synthi 100, probar y experimentar con ciertos módulos, y preguntar muchas dudas ante la ausencia de manuales detallados de la época.


