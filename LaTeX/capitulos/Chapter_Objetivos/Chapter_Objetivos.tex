\chapter{Objetivos}
\section{Objetivos generales}


\begin{enumerate}
	\item Promover el conocimiento de la síntesis sonora analógica, como medio para comprender las nuevas técnicas digitales.
	\item Revalorizar y dar a conocer el patrimonio del arte sonoro español, tangible e intangible, en muchos casos relegado a su almacenamiento, con vistas a fomentar su restauración y puesta a punto y publicación.
	\item Fomentar el uso del software libre tanto en el ámbito técnico como artístico, como medio idóneo de transmisión y creación de conocimiento.
	\item Contribuir a la proliferación de herramientas pedagógicas libres y colaborativas en el ámbito universitario y profesional.
	\item Poner a disposición de la comunidad educativa en general, y del GME de Cuenca en particular, una herramienta didáctica y de promoción del patrimonio instrumental del arte sonoro.
	\item Alentar la formación multidisciplinar y multicompetencial en el ámbito de la creación artística.
\end{enumerate}





\section{Objetivos específicos}

Dividimos los objetivos específicos en diferentes bloques. En cuanto a la creación de la aplicación informática, se destacan los siguientes:

\begin{enumerate}
	
	\item Diseñar y crear un programa informático que emule la interfaz y el funcionamiento del sintetizador EMS Synthi 100 del Gabinete de Música Electroacústica de Cuenca.
	\item Poner el foco en el uso pedagógico y artístico.
	\item Conferirle una arquitectura escalable con vistas a constantes mejoras y ampliaciones.
	\item Publicarse como software libre en los medios habituales de este momento.
	\item Implementar una interfaz de usuario abierta al uso de otros dispositivos tanto de software como de hardware.

\end{enumerate}

La presente memoria sobre diseño y programación de esta herramienta de software tiene por objetivos:

\begin{enumerate}
	\setcounter{enumi}{5}
	\item Ofrecer una descripción del funcionamiento de los principales módulos del sintetizador, atendiendo tanto a las similitudes como a las diferencias con versiones anteriores del mismo modelo.
	\item Servir de manual de la aplicación informática, sin perjuicio de la creación de los archivos de ayuda propios de la misma.
	\item Recoger información y contribuir a la escasa documentación sobre el funcionamiento del EMS Synthi 100 del GME.
\end{enumerate}



\begin{enumerate}
	\setcounter{enumi}{5}
	\item Poner a disposición de la comunidad educativa en general, y del GME de Cuenca en particular, una herramienta didáctica y de promoción del patrimonio instrumental del arte sonoro.
\end{enumerate}

