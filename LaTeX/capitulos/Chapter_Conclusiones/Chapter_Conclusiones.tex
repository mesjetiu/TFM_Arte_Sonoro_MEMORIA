\chapter[Conclusiones]{Conclusiones}
\chaptermark{Conclusiones}


Como final, una confesión. El primer objetivo de este trabajo ha sido el de aprender, y la motivación, la curiosidad. Este mismo motor es el que me ha movido intelectualmente en todas las direcciones por las que me he movido y me muevo en la vida. Y debe de haberse cumplido en el presente trabajo, a juzgar por tantos momentos de satisfacción que me ha entregado las incontables horas delante del ordenador, leyendo libros de síntesis analógica y digital o, de un modo muy especial, visitando el GME. 

Entre los objetivos expuestos al principio de la memoria (sección \ref{sec:objetivos}), se incluye el de aportar a la comunidad una herramienta fruto de mi trabajo. De nuevo he de decir, no sin cierta satisfacción, que en el camino recorrido he recibido más que he podido dar. Me he encontrado con personas muy generosas y de gran valía durante todo el máster, y, en particular, durante la elaboración del trabajo final. En la introducción reflexionaba en la importancia de la \textit{herramienta} en la producción artística, pero en mi conclusión quiero fijarme en quien hace el \textit{gesto} con ella. Al fin y al cabo, los artefactos y sus productos son de y para las personas. Detrás de un sintetizador, de una composición sonora, hay personas que crean y personas que escuchan.

Mi proyecto no es más que un grano de arena en la recuperación del GME de Cuenca y de todos sus fondos. Tanto si este trabajo es útil --ojalá-- como si no, espero al menos haber levantado polvo con él. Hay mucha gente trabajando porque el GME vuelva a la vida y ahora, de algún modo, me siento parte de ese gran equipo. 



