El presente trabajo describe el desarrollo del software <<\appName>>, un emulador digital del sintetizador modular EMS Synthi 100 del Gabinete de Música Electroacústica de Cuenca. Este software está implementado en el lenguaje de SuperCollider (\textit{sclang}), y su destino es el de ser una herramienta pedagógica para la difusión y el conocimiento de este sintetizador analógico, en actual proceso de restauración y con un importante hueco en la historia de la música electroacústica mundial. Lanzado bajo una licencia de software libre, está a disposición de la comunidad para cualquier propósito musical o de estudio. En la memoria se recoge una descripción pormenorizada del funcionamiento de los módulos emulados, así como cada una de las decisiones de diseño tomadas.
\\[1.5cm]
\palabrasClave{sintetizador, emulador, electroacústica, SuperCollider, música con ordenador}