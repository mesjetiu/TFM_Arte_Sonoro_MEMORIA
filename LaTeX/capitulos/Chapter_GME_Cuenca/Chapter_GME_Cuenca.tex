\chapter[El Synthi 100 del GME de Cuenca]{El sintetizador EMS Synthi 100 del Gabinete de Música Electroacústica de Cuenca}
\chaptermark{El Synthi del GME}


\section{Breve historia del GME}
\sectionmark{Historia del GME}

El GME aparece en Cuenca integrado en todo un planteamiento de apuesta por el arte en general y el arte contemporáneo en particular. Éste nace con la intención de poner en el panorama nacional e internacional no sólo un conjunto de herramientas tecnológicas a disposición de los compositores, sino, con una clara dirección pedagógica y formativa, ser el primer centro público de enseñanza, creación, investigación y difusión de la Música Electroacústica en España \cite[p.~317]{GME}. De este modo se situaba en el contexto español con personalidad propia respecto a otros grandes centros de esta música coetáneos como el estudio Phonos de Barcelona, el Laboratorio de Música Electrónica de la Escuela de Música Jesús Guridi de Vitoria o el Laboratorio de Informática y Electrónica Musical del Centro para la Difusión de la Música Contemporánea (LIEM-CDMC), por citar unos de los más emblemáticos\footnote{Para una pequeña historia de la música electroacústica en España, ver Epílogo a la edición española de Andrés Lewin-Richter, en \citeNP{Supper}}.

Fue inaugurado en 1983, dirigido por Pablo López de Osaba, director por aquel entonces del Conservatorio Profesional de Música, el Museo de Arte Abstracto Español y la Semana de Música Religiosa de Cuenca. Nació sin compositor asociado, siendo su primer técnico Leopoldo Amigo (1983--1990). En 1989 se contrata a Gabriel Brnčić como profesor de <<Teoría de la Composición>> y <<Teoría y Práctica de la Música con Medios Electroacústicos e Informáticos>>. Su labor se prolongará hasta 1998. Entre 1990 y 2006, el trabajo de Leopoldo Amigo será relevad por el del compositor Julio Sanz Vázquez. 

Los años en los que Gabriel ejerció su labor musical y magisterio coincidieron con los de mayor florecimiento de esta institución. Se organizaron los primeros cursos desde la oficialidad de un Conservatorio Profesional en España especializados en música electroacústica, así como conciertos mensuales en diversos espacios conquenses. Por sus instalaciones han pasado los más importantes compositores del momento, tanto por invitación como por iniciativa propia, para crear, dar conferencias o estrenar obras.

Tras la escisión del contrato de Gabriel Brnčić en 1998, el GME entró en un periodo de decadencia institucional por motivos políticos (que no son objeto de este trabajo) que lo haría desaparecer a pesar de los esfuerzos llevados a cabo por su técnico Julio Sanz. Desde septiembre de 2006, momento en el que sus instalaciones fueron desmanteladas, un gran trabajo de recuperación y catalogación llega hasta nuestros días de la mano de la asociación AVADI (Audio Vídeo Arte Digital Interactivo), que nace con el objetivo de preservar el ingente material creado en el GME y luchar por la continuidad de la institución de cara al futuro. 


\section{EMS y el Synthi 100}
\sectionmark{EMS y el Synthi 100}