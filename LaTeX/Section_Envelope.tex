\section{Envelope Shapers}

El <<generador de envolvente>> (conocido por sus nombres en inglés, \textit{Envelope Shaper} o \textit{Envelope Generator}), es uno de los módulos, junto con el de los osciladores, que no suelen faltar en los sintetizadores analógicos. Pero esta omnipresencia no es paralela a su estandarización. Los parámetros que controlan el generador de envolvente dependen de cada fabricante e, incluso, del modelo. Más aún, una de las grandes diferencias entre el Synthi 100 de Cuenca y el resto de Synthi 100 son los parámetros que este permite variar, lo cual da fe de la velocidad a la que el concepto de envolvente se desarrollaba por aquellos años.

\subsection{Un poco de teoría...}

Una <<envolvente>> puede ser descrita como la forma que se da al desarrollo del nivel de un parámetro de una señal, habitualmente la amplitud. De esta forma, una señal continua generada, por ejemplo, con un oscilador, puede adquirir una evolución dinámica, así como un inicio y un final. Puesto que una envolvente puede <<envolver>> a cualquier tipo de señal, sea esta de audio como de voltaje, podemos hacer variar con ella otros parámetros como la frecuencia de un oscilador, la frecuencia de corte de un filtro, el nivel de salida de otra envolvente, etc.

Históricamente, la envolvente surge inspirada por el comportamiento físico de los instrumentos musicales (imagen de la forma de onda de una nota de un piano).....

Parámetros básicos..., diferentes tipos de envolvente.

\subsection{El generador de envolventes del Synthi 100}

Diferencias entre la envolvente del Synthi 100 original y el de Cuenca.

Comportamiento anómalo y los compromisos adoptados en el emulador.

(Crear tabla con las diferentes envolventes, las del Synthi 100, las del Synthi 100 de Cuenca y el emulador.)